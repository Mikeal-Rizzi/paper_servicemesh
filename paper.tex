\documentclass[conference]{IEEEtran}
\IEEEoverridecommandlockouts

\usepackage{color}
\usepackage{url}
\usepackage{graphicx}
\usepackage{subfigure}
\usepackage{booktabs}
\usepackage{multirow}
\usepackage{balance}
\usepackage{pifont}
\usepackage[colorlinks,linkcolor=black,urlcolor=black]{hyperref}
\usepackage{amsmath,amssymb,amsfonts}

\graphicspath{{figures/}}
\newcommand{\figref}[1]{Fig.~\ref{#1}}
\newcommand{\tabref}[1]{Table~\ref{#1}}
\newcommand{\secref}[1]{Section~\ref{#1}}

\newcommand{\dong}[1]{\textcolor{blue}{[dong: #1]}}
\newcommand{\libin}[1]{\textcolor{red}{[libin: #1]}}
\newcommand{\zhao}[1]{\textcolor{green}{[zhao: #1]}}

\newcommand{\cmark}{\ding{52}}
\newcommand{\xmark}{\ding{56}}

\begin{document}

\title{Service Mesh: A Simple Survey}

\iffalse
\author{\IEEEauthorblockN{1\textsuperscript{st} Given Name Surname}
  \IEEEauthorblockA{\textit{dept. name of organization (of Aff.)} \\
  \textit{name of organization (of Aff.)}\\
  City, Country \\
  email address}
\and
  \IEEEauthorblockN{2\textsuperscript{nd} Given Name Surname}
  \IEEEauthorblockA{\textit{dept. name of organization (of Aff.)} \\
  \textit{name of organization (of Aff.)}\\
  City, Country \\
  email address}
\and
  \IEEEauthorblockN{3\textsuperscript{rd} Given Name Surname}
  \IEEEauthorblockA{\textit{dept. name of organization (of Aff.)} \\
  \textit{name of organization (of Aff.)}\\
  City, Country \\
  email address}
}
\fi

\maketitle

\begin{abstract}
Microservice architecture is widely used by Internet companies nowadays.
It is designed for the complex services.
Developers can easily deploy and manage a single service.
However, it also raise the complexity of the whole system, which makes the communication components between services bloated.
Service mesh is introduced to improve the management of the large amounts of microserivces.
Regardless of the design and implement of the single service, service mesh can work directly on the container of the service with the help of container technology, e.g. Docker.
In addition to manage the network communication, service mesh also have the various abilities, e.g., service registration and service discovery, of the tradditional microservices architecture.
We presant this paper to make a simple survey on service mesh, to introduce its history and applications.
Besides, we concretely introduce a popular architecture, Istio, to help better understand the principle of service mesh.
\cite{li2019service}
\end{abstract}

\begin{IEEEkeywords}
  Microservices, Service Mesh, Service Oriented Architecture
\end{IEEEkeywords}

\section{Introduction} \label{sec:intro}
\cite{li2019service}

\section{Service Mesh} \label{sec:mesh}

\section{Istio} \label{sec:istio}

\section{Applications} \label{sec:app}

\section{Conclusion}

\bibliographystyle{IEEEtran}
\bibliography{mybib}

\end{document}